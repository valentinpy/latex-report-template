%%=====================================================================================
% Settings
%%=====================================================================================
\newcommand{\mypapersize}{A4}
\newcommand{\mylaterality}{oneside} %% "oneside" or "twoside"
\newcommand{\myparskip}{half}
\newcommand{\myfontsize}{11pt}
\newcommand{\mylinespread}{singlespacing} % e.g.onehalfspacing, doublespacing, singlespacing
\newcommand{\mylanguage}{french} %american

\documentclass[%
  fontsize=\myfontsize,
	paper=\mypapersize,
	parskip=\myparskip,
	DIV=calc,
	headinclude=true,
	footinclude=true,
	open=right,
	appendixprefix=true,	% include appendix?
	bibliography=totoc,		% include an unnumbered unit of bibliography to the table of contents
	BCOR=10mm,        	  % binding correction (depends on how you bind
	\mylaterality,        % alternative: twoside
  \mylanguage
]{scrbook}

%Encoding
\usepackage[utf8]{inputenc}
\usepackage[T1]{fontenc}

% language adaptions
\usepackage[\mylanguage]{babel}

%% general metadata:
\newcommand{\myauthor}{Valentin Py}  %% also used for PDF metadata (hyperref)
\newcommand{\mydocumentsubject}{Branche }
\newcommand{\mysubject}{Nom du projet}  %% also used for PDF metadata (hyperref)
\newcommand{\myshortsubject}{Nom du projet raccoruci}  %% also used for PDF metadata (hyperref)
\newcommand{\mykeywords}{Consommation, Energie, TLS, Cryptographie}  %% also used for PDF metadata (hyperref)

%% this information is used only for generating the title page:
\newcommand{\myuniversity}{HES-SO Master} %% your university/school
\newcommand{\myfaculty}{HE-Arc} %% your university/school
\newcommand{\myinstitutehead}{Prof 1}
\newcommand{\mysupervisor}{Prof 2}

%%=====================================================================================
% Colors
%%=====================================================================================
% includes some colors
\usepackage[table,usenames,dvipsnames]{xcolor}

% Define some colors used in the template
\definecolor{grey}{gray}{0.95}
\definecolor{dkgreen}{rgb}{0,0.6,0}

\definecolor{mauve}{rgb}{0.58,0,0.82}
\definecolor{red}{rgb}{1,0,0}
\definecolor{DispositionColor}{RGB}{0, 0, 0}



%%=====================================================================================
%% includes pictures --> possible to include them in header/footer
%%=====================================================================================
\usepackage[pdftex]{graphicx}


%%=====================================================================================
%% Style
%%=====================================================================================
% override some list packages that they use babel -> see koma script doc
\usepackage{scrhack}

% nicer quotes
\usepackage[%
  autostyle,          % adapts language setting
  strict,             % turns warnings into errors
  english=american    % use american quotes style
]{csquotes}

\usepackage[automark]{scrpage2}							% allows usage of header and footer
\usepackage[perpage, hang]{footmisc} 				% footnote options

\renewcommand{\headfont}{\normalfont\sffamily\color{DispositionColor}}
\renewcommand{\pnumfont}{\normalfont\sffamily\color{DispositionColor}}
\addtokomafont{disposition}{\color{DispositionColor}}
\addtokomafont{caption}{\color{DispositionColor}\footnotesize}
\addtokomafont{captionlabel}{\color{DispositionColor}}

\usepackage{calc} %% used for calculation of header footer etc. ...

%% change page layout
\addtolength{\oddsidemargin}{-.875in}
\addtolength{\evensidemargin}{-.875in}
\addtolength{\textwidth}{1.75in}
\addtolength{\topmargin}{-.875in}
\addtolength{\textheight}{3in}  %1.75

% header and footer
\pagestyle{scrheadings}							% for customization for header and footer
\renewcommand{\chapterpagestyle}{scrheadings}	% include header and footer on chapter pages
\clearscrheadfoot

% http://ctan.uib.no/macros/latex/contrib/koma-script/doc/scrguien.pdf
% page 204
\ihead{\myshortsubject}
\ifoot{\vspace{-0.25cm}\myauthor}
\ofoot{ \vspace{-0.25cm} \thepage}
\automark{chapter}
\setheadsepline[ \textwidth + 5pt ]{1pt} % seperation line for header...
\setfootsepline[ \textwidth + 5pt ]{1pt} % ... and footer
\setkomafont{footsepline}{\color{red}} 	 % change colors of seperation lines
\setkomafont{headsepline}{\color{red}}

%%=====================================================================================
%% Table on Contents
%%=====================================================================================
\setcounter{tocdepth}{1}
\setcounter{secnumdepth}{1}

\usepackage{makecell}

\usepackage{tabulary}
\newcolumntype{K}[1]{>{\centering\arraybackslash}p{#1}}

\usepackage{verbatim}


%%=====================================================================================
%% bibliography with biber/biblatex
%%=====================================================================================
\usepackage[backend=biber, %% using "biber" to compile references (instead of "biblatex")
            style=numeric, %% see biblatex documentation
            style=alphabetic, %% see biblatex documentation
            backref=true, %% create backlings from references to citations
            natbib=true, %% offering natbib-compatible commands
            hyperref=true,
            sorting=none, %% using hyperref-package references
]{biblatex}

\addbibresource{bib/bibliography.bib}


%%=====================================================================================
%% SIUNITSX -- simplified usage of SI-units
%%=====================================================================================
\usepackage[
            group-digits=false,
            exponent-product = \cdot, % use \cdot instead * for exponent product
            binary-units=true,
            load-configurations=binary,
            load-configurations=abbreviations,
]{siunitx}

% Easy typesetting hex and bin and oct
\usepackage[autolanguage, nosepfour]{numprint}
\usepackage{nbaseprt}

%%=====================================================================================
%% ifthen and todonotes puts to-do-notes in the printed document if you want
%%=====================================================================================
%% used to disable todonotes package
\usepackage{ifthen}
\newboolean{myaddlistoftodos}

\reversemarginpar %To display the todo in the right margin
\setlength{\marginparwidth}{2cm} %To correct the bad placement
\makeatletter\let\chapter\@undefined\makeatother %this line force to consider the last todo in the list of todo. Without that, the last todo is missing.

\usepackage[english]{todonotes}

%%=====================================================================================
%% Tables, figures, enums, etc...
%%=====================================================================================
%% nice rule's for tables try \toprule \midrule \bottomrule
\usepackage{booktabs}

%% set width of table and more
\usepackage{tabularx}										% creates tables

%% rotate tables and figures
\usepackage{rotating}

%% define caption style
\addtokomafont{caption}{\small} 	% small captions
\usepackage[font=small, width=0.9\textwidth, format=plain, labelfont=bf]{caption}

%%
\usepackage{subfigure}
\usepackage{placeins}

%% customize item look
\usepackage{enumitem}

%%=====================================================================================
%% some utility stuff
%%=====================================================================================
% improved typographical settings
\usepackage[%
    protrusion=true, %
    factor=900       %
]{microtype}

%% switch of extra space after punctuation
\frenchspacing

%% switches to Palatino with small caps and old style figures
\usepackage[%
            sc,%
            osf,%
]{mathpazo}

%% kills space between items
\setlist{noitemsep}

%doc% For additional special characters available by \verb#\ding{}#
\usepackage{pifont}  %% Sonderzeichen fuer Titelseite \ding{}

%doc% This package is required for intelligent spacing after commands
\usepackage{xspace}

%doc% This package offers strikethrough command \verb+\sout{foobar}+.
\usepackage[normalem]{ulem}

%% prevent club & widow penalty
\clubpenalty10000
\widowpenalty10000
\displaywidowpenalty10000




%%=====================================================================================
%% lslistening: used to include source code
%%=====================================================================================
\usepackage{listings}				% include source code

\lstset{% 							% options for representation of source code
  backgroundcolor=\color{grey},   % choose the background color; you must add \usepackage{color} or \usepackage{xcolor}
  basicstyle=\footnotesize,        % the size of the fonts that are used for the code
  breakatwhitespace=false,         % sets if automatic breaks should only happen at whitespace
  breaklines=true,                 % sets automatic line breaking
  captionpos=b,                    % ses the caption-position to bottom
  commentstyle=\color{dkgreen}, % comment style
  deletekeywords={...},            % if you want to delete keywords from the given language
  escapeinside={\%*}{*)},          % if you want to add LaTeX within your code
  frame=lines,                     % adds a frame around the code
  keepspaces=true,                 % keeps spaces in text, useful for keeping indentation of code (possibly needs columns=flexible)
  keywordstyle=\color{blue},       % keyword style
  language=C,                      % the language of the code
  morekeywords={*,...},            % if you want to add more keywords to the set
  numbers=left,                    % where to put the line-numbers; possible values are (none, left, right)
  numbersep=8pt,                   % how far the line-numbers are from the code
  numberstyle=\tiny\color{gray},   % the style that is used for the line-numbers
  rulecolor=\color{black},         % if not set, the frame-color may be changed on line-breaks within not-black text (e.g. comments (green here))
  showspaces=false,                % show spaces everywhere adding particular underscores; it overrides 'showstringspaces'
  showstringspaces=false,          % underline spaces within strings only
  showtabs=false,                  % show tabs within strings adding particular underscores
  stepnumber=1,                    % the step between two line-numbers. If it's 1, each line will be numbered
  stringstyle=\color{blue},        % string literal style
  tabsize=2,                       % sets default tabsize to 2 spaces
  title=\lstname,                  % show the filename of files included with \lstinputlisting; also try caption instead of title
}

%theme for C
\lstdefinestyle{C}{
	language=c, showspaces=false,
	keywordstyle=		\color{blue}\scriptsize,
	commentstyle=	  \color{dkgreen}\scriptsize\sffamily,
	stringstyle=		\color{mauve}\scriptsize,
	title=\lstname,
	escapeinside={//(*}{*)},
	morekeywords={},
	numbers = left, numberstyle=\tiny\color{black}
}

%%=====================================================================================
%% pdfcompresslevel from 0 to 10; std is fine
%%=====================================================================================
\pdfcompresslevel=9

%%=====================================================================================
%% Hyperref should always be the last package added --
%%=====================================================================================
\usepackage[%							% enables typesettings for hyperlinks
	pdftitle={\mysubject},%
	pdfauthor = {\myauthor},%
	pdfsubject = {\mysubject},%
  colorlinks={false},
  pdfcreator={pdfTex},%
  pdfkeywords={\mykeywords},%
  pdftex = true,%
  backref,%
  pagebackref=false, % creates backward references too
  bookmarks=True, %
  bookmarksopen=false, % when starting with AcrobatReader, the Bookmarkcolumn is opened
  pdfpagemode=UseOutlines,% None, UseOutlines, UseThumbs, FullScreen
  plainpages=false % correct, if pdflatex complains: ``destination with same identifier already exists''
]{hyperref}								% should be the last package to be inlcuded!
